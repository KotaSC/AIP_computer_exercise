\documentclass{jarticle}

\usepackage[dvipdfmx]{graphicx}
\usepackage{listings,listing}
\usepackage{amsmath,amssymb}

\lstset{
  basicstyle={\ttfamily},
  identifierstyle={\small},
  commentstyle={\smallitshape},
  keywordstyle={\small\bfseries},
  ndkeywordstyle={\small},
  stringstyle={\small\ttfamily},
  frame={tb},
  breaklines=true,
  columns=[l]{fullflexible},
  numbers=left,
  xrightmargin=0zw,
  xleftmargin=3zw,
  numberstyle={\scriptsize},
  stepnumber=1,
  numbersep=1zw,
  lineskip=-0.5ex
}

\setlength{\evensidemargin}{1.0cm}
\setlength{\oddsidemargin}{1.0cm}
\setlength{\textwidth}{14.0cm}
\setlength{\textheight}{23.0cm}

\begin{document}
\begin{titlepage}
\title{Computer exercise(5/9)}
\author{
  Student number:661219001190   \\
  Name:Kota Kawakami
}
\end{titlepage}

\maketitle

\section{Method and algorithm}

{\bf [Step1]} \\

Randomly select 10 points from each class as training data 

\begin{equation}
  \mathbf{x}_{\mathbf{k}}^{\mathrm{n}}(k=1,2, \ldots, 5 ; n=1,2, . ., 10)
\end{equation}\\



{\bf [Step2]} \\

Calculating mean vector $\mu_{k}$ and covariance matrix $\Sigma_{k} $ for each class 
using its training data $x_{k}^{n}$.\\

{\bf [Step3]} \\

Read target image and
$\mathbf{x}=[R(i . i) \quad G(i . j) \quad B(i . j)]^{T}$ for pixel(i, j)\\
 

{\bf [Step4]}\\

Calculate likelihood $p(\mathbf{x}|k)$ for all classes $(k=1, 2,.., 5)$.
  
\begin{equation}
    p(\mathbf{x}|k)
    =\frac{1}{(2 \pi)^{\frac{d}{2}} \operatorname{det}(\boldsymbol{\Sigma_{k}})^{\frac{1}{2}}} 
    \exp \left(-\frac{1}{2}(\mathbf{x}-\boldsymbol{\mu_{k}})^{\mathrm{T}} 
    \boldsymbol{\Sigma_{k}}^{-1}(\mathbf{x}-\boldsymbol{\mu_{k}})\right)
  \end{equation}\\

{\bf [Step5]}\\

Classify the pixel (i,j) to class c, if $p(\mathbf{x}|c) = {\rm max} \ p(\mathbf{x}|k)$ .\\

{\bf [Step6]}\\

Repeat step 3-5 for all pixels.

\section{Experimental result}

\begin{center}
  \includegraphics[width=10cm]{../image/target.png} \\
  Target image
\end{center}

\newpage

\begin{figure}[htbp]
  \begin{minipage}{0.3\hsize}
   \begin{center}
    \includegraphics[width=20mm]{../image/class1.eps} \\
    class1
   \end{center}
  \end{minipage}
  \begin{minipage}{0.3\hsize}
   \begin{center}
    \includegraphics[width=20mm]{../image/class2.eps} \\
    class2
   \end{center}
  \end{minipage}
  \begin{minipage}{0.3\hsize}
    \begin{center}
     \includegraphics[width=20mm]{../image/class3.eps} \\
     class3
    \end{center}
   \end{minipage}
 \end{figure}
 
 \begin{figure}[htbp]
  \begin{minipage}{0.3\hsize}
   \begin{center}
    \includegraphics[width=3mm]{../image/class4.eps} \\
    class4
   \end{center}
  \end{minipage}
  \begin{minipage}{0.3\hsize}
   \begin{center}
    \includegraphics[width=20mm]{../image/class5.eps} \\
    class5
   \end{center}
  \end{minipage}
 \end{figure}

 \begin{center}
  \includegraphics[width=15cm]{../image/0509.png} \\
  Result image
\end{center}

\section{Program list}

\begin{lstlisting}[basicstyle=\ttfamily\footnotesize, frame=single]
  import numpy as np
  from PIL import Image
  import random
  import math
  from matplotlib import pyplot as plt
  
  def main():
  
      # Number of class
      d         = 5
  
      # Number of pixels randomly selected from training data
      train_num = 10
  
      # [R, G, B] channel
      channel   = 3
  
      mean      = np.empty( (d, channel), float )
      cov       = np.empty( (channel, channel, d), float )
  
      for i in range(d):
  
          # Read each class image
          img                 = np.array(Image.open('image/class' + str(i+1) + '.bmp'))
          height, width       = img.shape[0], img.shape[1]
  
          sum_training_data   = np.zeros(3)
          sum_training_data2  = np.zeros(3)
  
          mean_training_data  = np.zeros(3)
          mean_training_data2 = np.zeros(3)
  
          tmp_R = []
          tmp_G = []
          tmp_B = []
  
          for r in range(train_num):
  
              # Randomly select pixels from training data
              tmp                 = np.array(img[random.randint(0, height-1), random.randint(0, width-1)], dtype='int64')
  
              sum_training_data  += tmp
              sum_training_data2 += tmp**2
  
              tmp_R.append(tmp[0])
              tmp_G.append(tmp[1])
              tmp_B.append(tmp[2])
  
          # Caluculate mean vector
          mean_training_data  = sum_training_data/10
          mean_training_data2 = sum_training_data2/10
  
          # Cluculate variance
          var = mean_training_data2 - mean_training_data**2
  
          sum_RG = 0
          sum_GB = 0
          sum_BR = 0
  
          for j in range(train_num):
  
              sum_RG += (tmp_R[j] - mean_training_data[0])*(tmp_G[j] - mean_training_data[1])
              sum_GB += (tmp_G[j] - mean_training_data[1])*(tmp_B[j] - mean_training_data[2])
              sum_BR += (tmp_B[j] - mean_training_data[2])*(tmp_R[j] - mean_training_data[0])
  
          # Caluculate covariance
          cov_RG = sum_RG/train_num
          cov_GB = sum_GB/train_num
          cov_BR = sum_BR/train_num
  
          # Caluculate covariance matrix
          cov_Matrix = np.array([var[0], cov_RG, cov_BR, cov_RG, var[1], cov_GB, cov_BR, cov_GB, var[2]]).reshape((3,3))
  
          mean[i, :]   = mean_training_data
          cov[:, :, i] = cov_Matrix
  
      # Read target image
      satellite_image = np.array(Image.open('image/irabu_zhang1.bmp'))
      height, width   = satellite_image.shape[0], satellite_image.shape[1]
      result_image    = np.empty( (height, width), dtype='int64' )
  
      for h in range(height):
          for w in range(width):
  
              # Each pixle of target image
              x = satellite_image[h,w,:]
  
              # Caluculate likelihood for all classes
              P = []
              for k in range(d):
  
                  # Caluculate determinant
                  det = np.linalg.det(cov[:,:,k])
  
                  # Caluculate inverse matrix
                  inv = np.linalg.inv(cov[:,:,k])
  
                  left  = ( 2*np.pi**(d/2) * det**(0.5) ) ** (-1)
                  D     = np.dot( np.dot( (x-mean[k,:]).T, inv ), x-mean[k,:] )
                  right = np.exp(-0.5*D)
                  L     = left * right
  
                  P.append(L)
  
              result_image[h,w] = np.argmax(P) + 1
  
      plt.figure(figsize=(8, 6))
      plt.imshow(result_image, cmap='jet')
      plt.colorbar(shrink=0.85)
      plt.savefig("0509.png")
      plt.show()
  
  if __name__ == "__main__":
      main()
\end{lstlisting}

\end{document}
